% Document setup
\documentclass[article, a4paper, 11pt, oneside]{memoir}
\usepackage[utf8]{inputenc}
\usepackage[T1]{fontenc}
\usepackage[UKenglish]{babel}

% Document info
\newcommand\doctitle{Gray codes}
\newcommand\docauthor{Danny Nygård Hansen}

% Formatting and layout
\usepackage[autostyle]{csquotes}
\renewcommand{\mktextelp}{(\textellipsis\unkern)}
\usepackage[final]{microtype}
\usepackage{xcolor}
\frenchspacing
\usepackage{latex-sty/articlepagestyle}
\usepackage{latex-sty/articlesectionstyle}

% Fonts
\usepackage{amssymb}
\usepackage[largesmallcaps,partialup]{kpfonts}
\DeclareSymbolFontAlphabet{\mathrm}{operators} % https://tex.stackexchange.com/questions/40874/kpfonts-siunitx-and-math-alphabets
\linespread{1.06}
% \let\mathfrak\undefined
% \usepackage{eufrak}
\DeclareMathAlphabet\mathfrak{U}{euf}{m}{n}
\SetMathAlphabet\mathfrak{bold}{U}{euf}{b}{n}
% https://tex.stackexchange.com/questions/13815/kpfonts-with-eufrak
\usepackage{inconsolata}

% Hyperlinks
\usepackage{hyperref}
\definecolor{linkcolor}{HTML}{4f4fa3}
\hypersetup{%
	pdftitle=\doctitle,
	pdfauthor=\docauthor,
	colorlinks,
	linkcolor=linkcolor,
	citecolor=linkcolor,
	urlcolor=linkcolor,
	bookmarksnumbered=true
}

% Equation numbering
\numberwithin{equation}{chapter}

% Footnotes
\footmarkstyle{\textsuperscript{#1}\hspace{0.25em}}

% Mathematics
\usepackage{latex-sty/basicmathcommands}
\usepackage{latex-sty/framedtheorems}
\usepackage{tikz-cd}
\tikzcdset{arrow style=math font} % https://tex.stackexchange.com/questions/300352/equalities-look-broken-with-tikz-cd-and-math-font
\usetikzlibrary{babel}

% Lists
\usepackage{enumitem}
\setenumerate[0]{label=\normalfont(\arabic*)}

% Bibliography
\usepackage[backend=biber, style=authoryear, maxcitenames=2, useprefix]{biblatex}
\addbibresource{references.bib}

% Title
\title{\doctitle}
\author{\docauthor}

\newcommand{\calM}{\mathcal{M}}
\newcommand{\calL}{\mathcal{L}}
\newcommand{\calV}{\mathcal{V}}
\newcommand{\calW}{\mathcal{W}}
\newcommand{\calU}{\mathcal{U}}
\newcommand{\calE}{\mathcal{E}}
\DeclareMathOperator{\sign}{sgn}
\DeclareMathOperator{\adj}{adj}
\DeclareMathOperator{\spec}{Spec}
\DeclareMathOperator{\trace}{tr}


\usepackage{listofitems}
\setsepchar{,}

\makeatletter

\newcommand{\mat@dims}[1]{%
    \readlist*\@dims{#1}%
    \ifnum \@dimslen=1
        \def\@dimsout{\@dims[1]}%
    \else
        \def\@dimsout{\@dims[1], \@dims[2]}%
    \fi
    \@dimsout
}
\makeatother

\newcommand{\matgroup}[3]{\mathrm{#1}_{#2}(#3)}
\newcommand{\matGL}[2]{\matgroup{GL}{#1}{#2}}
\newcommand{\trans}{^{\top}}
% \newcommand{\mat}[2]{\calM_{\mat@dims{#1}}(#2)}
\newcommand{\mat}[2]{\mathrm{Mat}_{\mat@dims{#1}}(#2)}
\newcommand{\matO}[1]{\mathrm{O}(#1)}
\newcommand{\matSO}[1]{\mathrm{SO}(#1)}
\newcommand{\field}{\mathbb{F}}
\DeclareMathOperator{\chr}{char}

% Break
\usepackage{adforn}
\newcommand\fleuronbreak{\fancybreak{\textcolor{linkcolor}{\adfhangingflatleafleft}}}

\DeclarePairedDelimiter{\floor}{\lfloor}{\rfloor}
\newcommand{\bin}[2][]{\mathrm{bin}_{#1}(#2)}
\newcommand{\gray}[2][]{\mathrm{gr}_{#1}(#2)}
\newcommand{\xor}{\oplus}
\newcommand{\shiftr}{^{\gg}}
\newcommand{\conc}{\circ}
\newcommand{\emptystring}{\lambda}

% TODO: Fix numbering

\begin{document}

\maketitle

\noindent If $a$ and $b$ are binary strings of the same length, we denote the bitwise exclusive disjunction of $a$ and $b$ by $a \xor b$. We denote the concatenation of $a$ with $b$ either by $a \conc b$ or $ab$. Also, if $b$ is a binary string, denote by $b\shiftr$ the right logical shift of $b$, i.e. the string obtained by removing the rightmost bit of $b$ and appending a $0$ on the left of the result.

Let $n \in \naturals$. For a number $k \in \naturals$ with $k < 2^n$ we denote the $n$-bit binary representation of $k$ by $\bin[n]{k}$. Furthermore, we denote the $n$-bit Gray code for $k$ by $\gray[n]{k}$. By definition, $\gray[0]{0} = \emptystring$ and
%
\begin{equation*}
    \gray[n+1]{k}
        = \begin{cases}
            0 \conc \gray[n]{k},           & k < 2^n,    \\
            1 \conc \gray[n]{2^{n+1}-1-k}, & k \geq 2^n.
        \end{cases}
\end{equation*}
%
for all $n \in \naturals$ and $(n+1)$-bit numbers $k$. We claim the following:

\begin{proposition}
    Let $n \in \naturals$, and let $k \in \naturals$ be an $n$-bit number. Writing $\bin[n]{k} = b_{n-1} \cdots b_0$ we have $\gray[n]{k} = a_{n-1} \cdots a_0$, where $a_{n-1} = b_{n-1}$ and
    %
    \begin{equation}
        \label{eq:gray-consecutive-bits}
        a_i
            = b_{i+1} \xor b_i
    \end{equation}
    %
    for $i \in \{0, \ldots, n-2\}$. That is,
    %
    \begin{equation*}
        \gray[n]{k}
            = b_{n-1} (b_{n-1} \xor b_{n-2}) \cdots (b_1 \xor b_0).
    \end{equation*}
    %
    Conversely we have
    %
    \begin{equation*}
        b_i
            = a_i \xor \cdots \xor a_{n-1}.
    \end{equation*}
\end{proposition}
%
The formula \cref{eq:gray-consecutive-bits} also holds in the case $i = n-1$ if we let $b_n = 0$, i.e. we prepend zeros if necessary.

\begin{proof}
    If $n = 0$, then the claim is obvious since there are no $0$-bit numbers. Now let $k$ be an $(n+1)$-bit number, so that $k < 2^{n+1}$, and write $\bin[n+1]{k} = b_n \cdots b_0$. If $k < 2^n$, then $b_n = 0$ and $\gray[n+1]{k} = 0 \conc \gray[n]{k}$. By induction we have
    %
    \begin{align*}
        \gray[n]{k}
            &= b_{n-1} (b_{n-1} \xor b_{n-2}) \cdots (b_1 \xor b_0) \\
            &= (b_n \xor b_{n-1}) (b_{n-1} \xor b_{n-2}) \cdots (b_1 \xor b_0),
    \end{align*}
    %
    so it follows that
    %
    \begin{equation*}
        \gray[n+1]{k}
            = b_n \conc \gray[n]{k}
            = b_n (b_n \xor b_{n-1}) (b_{n-1} \xor b_{n-2}) \cdots (b_1 \xor b_0)
    \end{equation*}
    %
    as claimed. If instead $k \geq 2^n$, then $b_n = 1$. Writing $k = 2^n + r$ with $0 \leq r < 2^n$ we have $\bin[n]{r} = b_{n-1} \cdots b_0$. Now notice that $\bin[n]{2^n - 1 - r} = \conj{b}_{n-1} \cdots \conj{b}_0$ since
    %
    \begin{equation*}
        (\conj{b}_{n-1} \cdots \conj{b}_0)_2 + r + 1
            = (\conj{b}_{n-1} \cdots \conj{b}_0)_2 + (b_{n-1} \cdots b_0)_2 + 1
            = 2^n.
    \end{equation*}
    %
    By induction we have
    %
    \begin{align*}
        \gray[n]{2^n-1-r}
            &= \conj{b}_{n-1} (\conj{b}_{n-1} \xor \conj{b}_{n-2}) \cdots (\conj{b}_1 \xor \conj{b}_0) \\
            &= (b_n \xor b_{n-1}) (b_{n-1} \xor b_{n-2}) \cdots (b_1 \xor b_0)
    \end{align*}
    %
    since $b_n = 1$, so it follows that
    %
    \begin{align*}
        \gray[n+1]{k}
            &= b_n \conc \gray[n]{2^n - 1 - r} \\
            &= b_n (b_n \xor b_{n-1}) (b_{n-1} \xor b_{n-2}) \cdots (b_1 \xor b_0) 
    \end{align*}
    %
    as desired.

    For the final claim, simply notice that
    %
    \begin{align*}
        a_i \xor \cdots \xor a_{n-1}
            &= (b_i \xor b_{i+1}) \xor (b_{i+1} \xor b_{i+2}) \xor \cdots \xor (b_{n-2} \xor b_{n-1}) \xor b_{n-1} \\
            &= b_i \xor (b_{i+1} \xor b_{i+1}) \xor (b_{i+2} \xor \cdots \xor b_{n-2}) \xor (b_{n-1} \xor b_{n-1}) \\
            &= b_i.
    \end{align*}
    %
    Alternatively we may notice that \cref{eq:gray-consecutive-bits} defines a linear system of equations with coefficients in $\ints/2\ints$ and invert this.
\end{proof}



\begin{corollary}
    For $n \in \naturals$ and any $n$-bit number $k$, we have
    %
    \begin{equation*}
        \gray[n]{k}
            = \bin[n]{k} \xor \bin[n]{k}\shiftr.
    \end{equation*}
\end{corollary}

\begin{proof}
    Writing $\bin[n]{k} = b_{n-1} \cdots b_0$, the proposition implies that
    %
    \begin{align*}
        \gray[n]{k}
            &= b_{n-1} (b_{n-1} \xor b_{n-2}) \cdots (b_1 \xor b_0) \\
            &= (0 \xor b_{n-1}) (b_{n-1} \xor b_{n-2}) \cdots (b_1 \xor b_0).
    \end{align*}
    %
    But $\bin[n]{k}\shiftr = 0 b_{n-1} \cdots b_1$, so the claim follows.
\end{proof}


\nocite{*}

\printbibliography

\end{document}